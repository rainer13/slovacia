\documentclass[12pt]{report}
\usepackage[romanian]{babel}
\usepackage[utf8]{inputenc}
\usepackage{setspace}
\usepackage{fancyhdr}
\usepackage{amsmath}
\usepackage{amsfonts}
\usepackage{geometry}
\usepackage{amssymb}
\usepackage{graphicx}
\usepackage{makeidx}
\makeindex



\begin{document}
\begin{titlepage}
\setlength{\topsep}{0pt}
\vspace{-5cm}
\begin{center}

Universitatea de Vest din Timi\c soara\\
Facultatea  de  Matematic\u a  \c si  Informatic\u a\\Departamentul de Matematic\u a\\ Specializarea Matematic\u a
\end{center}

\vspace{6cm}
\begin{center}
\begin{huge}
\setstretch{0.5}
Lucrare de licent\u a\\
Spa\c tii Orlicz \c si aplica\c tii \^ in studiul comport\u arilor asimptotice ale sistemelor
\end{huge}
\end{center}
\vspace{2cm}
\begin{flushleft}
Coordonator \c stiin\c tific \hfill Autor\\
prof. univ. dr. Bogdan SASU \hfill Rainer RETZLER



\end{flushleft}

\vspace{3cm}
\begin{center}
Timi\c soara\\
2015
\end{center}
\end{titlepage}

\tableofcontents
\chapter{Introducere}
\paragraph{}
Teoria spa\c tiilor de func\c tii a inceput sa fie dezvoltata in jurul anului 1925 de catre A. N. Kolmogorv, care a incercat sa generalizeze notiunea de spatiu de functii dtabilit de spatiile ${L^p}$.
Munca acestuia a fost continuata A. Zygmund \c si E. C. Titchmarsch, care au descoperit spatiile Llog(L). O adevarata generalizare a fost pusa la punct in anul 1931 de Wldislaw Orlicz \c si Zygmund, care au folosit a\c sa-numitele func\c tii de tip Young (func\c tii cresc\u atoare convexe, nu peste tot nule sau infinite) pentru a reu\c si s\u a introduc\u a o norm\u a pe aceste spa\c tii.

\paragraph{}
Rezultatele acestora au fost mai t\^ arziu utilizate \^ in teoria controlului \c si studiului stabilit\u a\c tii asimptotice ale sistemelor ce pot fi modelate cu ajutorul ${C_0}$-emigrupurlor. Aceste compotante rezult\u a din teoreme de tipul Littman-Neerven.

\chapter{Spa\c tii de func\c tii}
\section{Norm\u a generalizat\u a de func\c tii}
\section{Spa\c tii vectoriale normate de func\c tii}
\section{Spa\c tii Banach de func\c tii}
\section{Clasele $\mathcal{B}(\mathbb{R})$ \c si $\mathcal{E}(\mathbb{R})$}
\chapter{Spa\c tii Orlicz}
\section{Func\c tia lui Young}
\section{Spa\c tii \c si norme Orlicz}
\section{Propriet\u a\c ti}
\section{Aplica\c tii}
\chapter{Concluzii}



\begin{thebibliography}{1}

\bibitem{}
  J. van Neerven;
  The Asymptotic Behaviour of Semigroups of linear Operators; Theory Advances and Applications, vol.88, Birkhauser 1996
\bibitem{}
  J. van Neerven;
  Exponential Stability Operators and Operator Semigroups; J. Funct. Anal., 130(1995)
  \bibitem{}
  M. Megan, B. Sasu, L. Sasu; Banach Function Spaces  and Stability of ${C_{0}}$ - semigroups; Sem. An. Mat. Apl. \^ in Teor. Controlului, nr. 90; 1998
  
  \bibitem{}
  Collin Bennett, Robert Sharpley; Interpolation of Operators; Academic Press Inc; London; 1988
  \bibitem{}
  Erwin Kreyszig;  Introductory Functional Analysis with Applications; John Wiley and Sons; New York; 1978
  \bibitem{}
  Walter Rudin; Functional Analysis, 2nd Edition; McGraw-Hill Inc.; New York; 1991

\end{thebibliography}




\end{document}