\documentclass[12pt,a4paper]{book}
\usepackage{amssymb}
\usepackage{eucal}
\usepackage{amsmath}
\newcommand{\p}{\partial}
\newcommand{\ds}{\displaystyle}


\newtheorem{thm}{\bf Teorema}[section]
\newtheorem{prop}{Propozi\c tia}[section]
\newtheorem{lema}{Lema}[section]
\newtheorem{de}{\bf Defini\c tia}[section]
\newtheorem{cons}{\bf Consecin\c ta}[section]
\newtheorem{rem}{\bf Remarc\u a}[section]
\newtheorem{ex}{Exerci\c tiul}[chapter]
\renewcommand{\contentsname}{Cuprins}
\renewcommand{\chaptername}{Capitolul}
\renewcommand{\bibname}{Bibliografie}


\begin{document}

\makeatletter

\section*{Introducere}
\vspace{10 mm}
\addcontentsline{toc}{section}{Introducere}
%\chapter*{Introducere}
%\markboth{Introducere}{}%
%\addcontentsline{toc}{chapter}{Introducere} \indent\indent Se consider\u a c\u a Analiza Func\c tional\u a s-a n\u ascut \^\i n iunie 1920, deoarece atunci St. Banach a depus la Universitatea Jan Kazimierz din Lvov (pe vremea aceea ora\c s \^\i n Polonia, azi \^\i n Ucraina) teza sa de doctorat "\emph{Sur les operations dans les ensembles abstraits et leur applications aux \' equations int\' egral}". \^ In aceast\u a tez\u a, care a fost publicat\u a \^\i n 1922, Banach a introdus, axiomatic, no\c tiunea de spa\c tiu normat complet, o no\c tiune general\u a \^\i n care, pentru prima dat\u a, au fost \^\i njghebate o structur\u a algebric\u a (cea de spa\c tiu liniar) \c si una topologic\u a (cea de spa\c tiu metric). \^ In 1928, Maurice Fr\' echet (1878 - 1973) a propus ca aceste spa\c tii normate complete s\u a se numeasc\u a spa\c tii Banach.

Ast\u azi, putem spune c\u a, studiul completitudinii spa\c tiilor normate, utilizate sau aflate la un moment dat \^\i n studiu, constituie o etap\u a fundamental a fi parcurs\u a \^\i naintea identific\u arii \c si a altor propriet\u a\c ti particulare ale acestor spa\c tii. Pe de o parte acest lucru se datoreaz\u a  \c si faptului c\u a studiul convergen\c tei unui \c sir, \^\i n prezen\c ta propriet\u a\c tii de completitudine, implic\u a numai utilizarea termenilor \c sirului nu \c si identificarea apriorii a "posibilei" limite a acestuia. Un alt argument ar fi faptul c\u a orice serie absolut convergent\u a (de vectori ai unui spa\c tiu normat complet) este \c si convergent\u a, iar studiul absolut convergen\c tei se reduce la cazul scalar. Nu ar fi lipsit de importan\c t\u a faptul c\u a, \^\i n prezen\c ta  propriet\u a\c tii de completitudine, sunt posibil a fi utilizate principiile de categorie ale Analizei Func\c tionale.

Lucrarea de fa\c t\u a se \^\i ncadreaz\u a \^\i n ideeea descris\u a de paragraful anterior, ea propun\^ andu-\c si a prezenta dou\u a clase speciale de spa\c tii normate complete \c si anume spa\c tiile Banach de func\c tii \c si spa\c tiile Banach de \c siruri. Aceste clase generalizeaz\u a  spa\c tiile de tip $L^p(X, \mathcal{A}, \mu)$, ($1 \leq p \leq \infty$) dar nu se reduc numai la acestea. Aceast\u a lucrare este structurat\u a pe patru capitole.

Primul capitol, "\emph{Spa\c tii normate. Spa\c tii Banach.}", are rolul de a \^\i ncadra tema lucr\u arii dar \c si de introducere a no\c tiunilor \c si rezultatelor generale de utilizate. Se bazeaz\u a \^\i n principal pe cursul de Analiz\u a Func\c tional\u a din anul III \c si a fost introdus aici pentru a-i conferii lucr\u arii un caracter auto-inclus. \^ In acest fel, cititorul poate parcurge lucrarea f\u ar\u a a face apel la surse exterioare.

Cel de-al doilea capitol, "\emph{Spa\c tii Banach de func\c tii}" introduce, \^\i ntr-o manier\u a abordabil\u a, \c si studieaz\u a, o clas\u a de spa\c tii Banach, cunoscute \^\i n Analiza Func\c tional\u a ca spa\c tii de func\c tii. Este strcturat\u a pe \c sapte sec\c tiuni (paragrafe) \c si are \^\i n special rolul de a prezenta spa\c tiile Orlicz \c si proprietatea de completitudine a acestora. Aceste spa\c tii Banach generalizeaz\u a spa\c tiile de tip $L^p$ al func\c tiilor absolut $p$-integrabile Lebesgue. Am folosit aici. \^\i n special, noti\c tele de la cursul sus\c tinut de D-na Conf. Dr. Adina Lumini\c ta Sasu la studen\c tii de la Master din anul \^\i nt\^ ai.

Al treilea capitol ala lucr\u arii, "\emph{Spa\c tii Banach de \c siruri}", este structurat pe dou\u a sec\c tiuni \c si prezint\u a  clasa de spa\c tiilor Banach de \c siruri, clas\u a din care fac parte \c si spa\c tiile de tip $\ell^p$ (spa\c tiul \c sirurilor scalare $p$-absolut sumabile). Sunt prezentate aici \c si spa\c tiile Banach de \c siruri invariante la transla\c tii cunoscute \c si sub de numirea de spa\c tii Sch\" affer.

Ultimul capitol al lucr\u arii, "\emph{Familii de evolu\c tie \^\i n timp discret. Dichotomie exponen\c tial uniform\u a}" este dedicat prezent\u arii unei aplica\c tii a spa\c tiilor Banach studiate \^\i n capitolele anterioare. Este astfel studiat\u a proprietatea de dichotomie a unei familii de evolu\c tie discret\u a \^\i n ipoteze de tip "intrare-ie\c sire" \^\i n care at\^ at spa\c tiul func\c tiilor de intrare c\^ at \c si cel de ie\c sire sunt spa\c tii de tip Sch\' affer.

\^ In \^\i ncheiere doresc s\u a mul\c tumesc d-nului Lect. Dr. Aurelian Craciunescu, coordonatorul \c stiin\c tific al acestei lucr\u ari de licen\c t\u a, pentru sprijinul acordat \^\i n elaborarea acesteia.
\end{document}